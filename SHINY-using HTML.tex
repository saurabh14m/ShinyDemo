
\documentclass[a4paper,12pt]{article}
%%%%%%%%%%%%%%%%%%%%%%%%%%%%%%%%%%%%%%%%%%%%%%%%%%%%%%%%%%%%%%%%%%%%%%%%%%%%%%%%%%%%%%%%%%%%%%%%%%%%%%%%%%%%%%%%%%%%%%%%%%%%%%%%%%%%%%%%%%%%%%%%%%%%%%%%%%%%%%%%%%%%%%%%%%%%%%%%%%%%%%%%%%%%%%%%%%%%%%%%%%%%%%%%%%%%%%%%%%%%%%%%%%%%%%%%%%%%%%%%%%%%%%%%%%%%
\usepackage{eurosym}
\usepackage{vmargin}
\usepackage{amsmath}
\usepackage{graphics}
\usepackage{epsfig}
\usepackage{subfigure}
\usepackage{fancyhdr}
%\usepackage{listings}
\usepackage{framed}
\usepackage{graphicx}

\setcounter{MaxMatrixCols}{10}
%TCIDATA{OutputFilter=LATEX.DLL}
%TCIDATA{Version=5.00.0.2570}
%TCIDATA{<META NAME="SaveForMode" CONTENT="1">}
%TCIDATA{LastRevised=Wednesday, February 23, 2011 13:24:34}
%TCIDATA{<META NAME="GraphicsSave" CONTENT="32">}
%TCIDATA{Language=American English}

\pagestyle{fancy}
\setmarginsrb{20mm}{0mm}{20mm}{25mm}{12mm}{11mm}{0mm}{11mm}
\lhead{Shiny} \rhead{Dublin \texttt{R}}
\chead{Using HTML}
%\input{tcilatex}
\begin{document}

\subsection*{Using HTML }

% Minimal HTML Interface

\begin{itemize}
\item Instead of ui.R, create a folder called \texttt{www} and place a file called \texttt{index.html}
inside this folder.
\item This is where you will define your interface.
\item See this examples :  \textit{github.com/rstudio/shiny-examples/tree/master/008-html}
\end{itemize}

\newpage
\begin{framed}
\begin{verbatim}
<html>

<head>
  <script src="shared/jquery.js" type="text/javascript"></script>
  <script src="shared/shiny.js" type="text/javascript"></script>
  <link rel="stylesheet" type="text/css" href="shared/shiny.css"/> 
</head>
 
<body>

  <h1>HTML UI</h1>
 
  <p>
    <label>Distribution type:</label><br />
    <select name="dist">
      <option value="norm">Normal</option>
      <option value="unif">Uniform</option>
      <option value="lnorm">Log-normal</option>
      <option value="exp">Exponential</option>
    </select> 
  </p>
 
  <p>
 
    <label>Number of observations:</label><br /> 
    <input type="number" name="n" value="500" min="1" max="1000" />

  </p>
 
  <pre id="summary" class="shiny-text-output"></pre> 
  
  <div id="plot" class="shiny-plot-output" 
       style="width: 100%; height: 400px"></div> 
  
  <div id="table" class="shiny-html-output"></div>
 
</body>
</html>
\end{verbatim}
\end{framed}
\end{document}